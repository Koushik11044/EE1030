%iffalse
\let\negmedspace\undefined
\let\negthickspace\undefined
\documentclass[journal,12pt,twocolumn]{IEEEtran}
\usepackage{cite}
\usepackage{amsmath,amssymb,amsfonts,amsthm}
\usepackage{algorithmic}
\usepackage{graphicx}
\usepackage{textcomp}
\usepackage{xcolor}
\usepackage{txfonts}
\usepackage{listings}
\usepackage{enumitem}
\usepackage{mathtools}
\usepackage{gensymb}
\usepackage{comment}
\usepackage[breaklinks=true]{hyperref}
\usepackage{tkz-euclide} 
\usepackage{listings}
\usepackage{gvv}                                        
\def\inputGnumericTable{}                                 
\usepackage[latin1]{inputenc}                                
\usepackage{color}                                            
\usepackage{array}                                             
\usepackage{longtable}                                       
\usepackage{calc}                                             
\usepackage{multirow}                                         
\usepackage{hhline}                                           
\usepackage{ifthen}                                           
\usepackage{lscape}

\newtheorem{theorem}{Theorem}[section]
\newtheorem{problem}{Problem}
\newtheorem{proposition}{Proposition}[section]
\newtheorem{lemma}{Lemma}[section]
\newtheorem{corollary}[theorem]{Corollary}
\newtheorem{example}{Example}[section]
\newtheorem{definition}[problem]{Definition}
\newcommand{\BEQA}{\begin{eqnarray}}
\newcommand{\EEQA}{\end{eqnarray}}
\newcommand{\define}{\stackrel{\triangle}{=}}
\theoremstyle{remark}
\newtheorem{rem}{Remark}
\begin{document}

\bibliographystyle{IEEEtran}
\vspace{3cm}

\title{Assignment-1}
\author{EE224BTECH11044 - Muthyala koushik$^{*}$% <-this % stops a space
}
\maketitle
\newpage
\bigskip

\renewcommand{\thefigure}{\theenumi}
\renewcommand{\thetable}{\theenumi}

\textbf{1.Section-B : JEE Main/AIEEE}
	

\begin{enumerate}[start=4]
	\item {\small If $f: \mathbb{R} \to \mathbb{R}$ satisfies $f(x+y) = f(x) + f(y)$ for all} $x, y \in \mathbb{R}$ and $f(1) = 7$, then \(\sum_{r=1}^n f(r)\) is \hfill \textbf{[2003]} %6th question 

    \begin{enumerate}
        \item \(\frac{7n(n+1)}{2}\)\\
        \item \(\frac{7n}{2}\)\\
        \item \(\frac{7(n+1)}{2}\)\\
        \item \(7n + (n+1)\)
    \end{enumerate}

    \item The function \( f \) from the set of natural numbers to integers is defined by

\[
f(n) = 
\begin{cases} 
\frac{n-1}{2} & \text{when } n \text{ is odd}, \\
-\frac{n}{2} & \text{when } n \text{ is even}.
\end{cases}
\] is

	\begin{enumerate}
		\item neither one-one nor onto
		\item one-one but not onto
		\item onto but not one-one
		\item one-one and onto both.
	\end{enumerate} 

       \item {\small The range of the function} $f(x)= 7-x\cdot P \cdot x-3$ \\ is \hfill \textbf{[2004]}
        \begin{enumerate}
		\item $\{1,2,3,4,5\}$
		\item $\{1,2,3,4,5,6\}$
		\item $\{1,2,3,4\}$
		\item $\{1,2,3\}$
	\end{enumerate}
       
\item {\small If $f:R \to S$, defined by $f(x)=sinx-\sqrt{3}cosx+1$},is onto,then the interval of S is	\hfill \textbf{[2004]}

	       \begin{enumerate}
		       \item $[-1,3]$
		       \item $[-1,1]$
		       \item $[0,1]$
		       \item $[0,3]$
	       \end{enumerate}
       \item The graph of the function y=f(x) is symmertrical about the line x=2,then \hfill \textbf{[2004]}
	       \begin{enumerate}
		       \item $f(x)=-f(-x)$
		       \item $f(2+x)=f(2-x)$
		       \item $f(x)=f(-x)$
		       \item $f(x+2)=f(x-2)$
	       \end{enumerate}
\pagebreak

 \item The domain of the function $f(x)$=\(\frac{sin^{-1}(x-3)}{\sqrt{9-x^{2}}}\) is\\
	 \hfill \textbf{[2004]}
              \begin{enumerate}
		      \item $[1,3]$
		      \item $[2,3)$
		      \item $[1,2]$
		      \item $[2,3]$
	      \end{enumerate}
     \item Let $f:(-1,1) \to B$,be a function defined by $f(x)=tan^{-1}\frac{2x}{1-x^{2}}$,then $f$ is both one-one and onto when $B$ is the interval \hfill \textbf{[2005]}
             \begin{enumerate}
		     \item $\left(0,\frac{\pi}{2}\right)$ \\
		     \item $\left[0,\frac{\pi}{2}\right)$ \\
		     \item $\left[-\frac{\pi}{2},\frac{\pi}{2}\right]$ \\
		     \item $\left(-\frac{\pi}{2},\frac{\pi}{2}\right)$
	     \end{enumerate}
     \item A function is macthed below against an interval where it is supposed to increasing.Which of the following pairs is incorrectly matcher? \hfill \textbf{[2005]}\\
	     \hspace{0.5cm}  \textbf{Interval} \hspace{1.4cm} \textbf{Function}
	     \begin{enumerate}
		     \item $(-\infty,\infty)$ \hspace{1cm} $x^{3}-3x^{2}+3x+3$ \\
		     \item $[2,\infty)$ \hspace{1cm} $2x^{3}-3x^{2}+3x+3$ \\
		     \item $\left(-\infty,\frac{1}{3}\right]$ \hspace{1cm} $3x^{2}-2x+1$ \\
		     \item $(-\infty,-4)$ \hspace{1cm} $x^{3}+6x^{2}+6$
	     \end{enumerate}
     \item A real valued function $f(x)$ satisfies the functional equation $$f(x-y)=f(x)f(y)-f(a-x)f(a+y)$$\\
	     where a given constant and $f(0)$=1,$f(2a-x)$ is equal to \hfill \textbf{[2005]}
	     \begin{enumerate}
		     \item $-f(x)$
		     \item $f(x)$
		     \item $f(a)+f(a-x)$
		     \item $f(-x)$
	     \end{enumerate}
	     \pagebreak
     \item {\small The Largest interval lying in $\left(-\frac{\pi}{2},\frac{\pi}{2}\right)$for which the function,
	     $f(x)=4^{-x^{2}}+cos^{-1}\left(\frac{x}{2}-1\right)+\log(cosx)$,is defined,is} \hfill \textbf{[2007]}
	     \begin{enumerate}
		     \item $\left[-\frac{\pi}{4},\frac{\pi}{2}\right)$ \\
		     \item $\left[0,\frac{\pi}{2}\right)$ \\
		     \item $[0,\pi]$ \\
		     \item $\left(-\frac{\pi}{2},\frac{\pi}{2}\right)$
	     \end{enumerate}
     \item Let $f:N\to Y$ be a function defined as $f(x)=4x+3$ where Y=$\{y\in \mathbb{N}:y=4x+3 for some x\in \mathbb{N}\}$.Show that $f$ is invertible and its inverse is \hspace{3cm} \hfill  \textbf{[2008]}
	     \begin{enumerate}
		     \item $g(y)=\frac{3y+4}{3}$ \\
		     \item $g(y)=4+\frac{y+3}{4}$ \\
		     \item $g(y)=\frac{y+3}{4}$ \\
		     \item $g(y)=\frac{y-3}{4}$
	     \end{enumerate}
     \item Let $f(x)=(x+1)^{2}-1,x\leq-1$\\
	     \textbf{Statement-1:}The set \{$x$:$f(x)$=$f^{-1}(x)$=\{0,-1\}\\
	     \textbf{Statement-2:}$f$ is a bijection.\hfill \textbf{[2009]}
	     \begin{enumerate}
		     \item {\small Statement-1 is true,Statement-2 is true.Statement}-2 is  not a correct explanation for Statement-1.
		     \item Statement-1 is true,Statement-2 is flase.
		     \item Statement-1 is false,Statement-2 is true.
		     \item {\small Statement-1 is true,Statement-2 is true.Statement}-2 is not a correct explanation for Statement-1.
	     \end{enumerate}
     \item For real $x$,let $f(x)=x^{3}+5x+1$,then \hfill \textbf{[2009]}
	     \begin{enumerate}
		     \item $f$ is onto $\mathbb{R}$ but not one-one
		     \item $f$ is one-one and onto $\mathbb{R}$
		     \item $f$ is neither one-one nor onto $\mathbb{R}$
		     \item $f$ is one-one but not onto $\mathbb{R}$
	     \end{enumerate}
     \item The domain of the function $f(x)=\frac{1}{\sqrt{|x|-x}}$ is \hfill \textbf{[2011]}\\
	     \begin{enumerate}
		     \item $(0,\infty)$
		     \item $(-\infty,0)$
		     \item $(-\infty,\infty)-\{0\} $
		     \item $(-\infty,\infty)$
	     \end{enumerate}
	     \pagebreak
     \item $x\in \mathbb{R}-\{0,1\}$,let $f_1(x)=\frac{1}{x}$,$f_2(x)=1-x$ and $f_3(x)=\frac{1}{1-x}$ be the three given functions.If a function,$J(X)$ satisfies ($f_2oJof_1$)($x$)=$f_3(x)$ then J(x) is equal to: \hfill \textbf{[JEE M 2019-9 Jan(M)]}
	     \begin{enumerate}
		     \item $f_3(x)$
		     \item $f_3(x)$
		     \item $f_2(x)$
		     \item $f_1(x)$
	     \end{enumerate}
	

\end{enumerate}


\end{document}
