%iffalse
\let\negmedspace\undefined
\let\negthickspace\undefined
\documentclass[journal,12pt,onecolumn]{IEEEtran}
\usepackage{cite}
\usepackage{amsmath,amssymb,amsfonts,amsthm}
\usepackage{algorithmic}
\usepackage{graphicx}
\usepackage{textcomp}
\usepackage{xcolor}
\usepackage{txfonts}
\usepackage{listings}
\usepackage{enumitem}
\usepackage{mathtools}
\usepackage{gensymb}
\usepackage{comment}
\usepackage[breaklinks=true]{hyperref}
\usepackage{tkz-euclide} 
\usepackage{listings}
\usepackage{gvv}                                        
\def\inputGnumericTable{}                                 
\usepackage[latin1]{inputenc}                                
\usepackage{color}                                            
\usepackage{array}                                             
\usepackage{longtable}                                       
\usepackage{calc}                                             
\usepackage{multirow}                                         
\usepackage{hhline}                                           
\usepackage{ifthen}    
\usepackage{float}                                     
\usepackage{lscape}
\usepackage{pgfplots}
\usepackage{multicol}
\usepackage{circuitikz}
\usetikzlibrary{patterns}

\newtheorem{theorem}{Theorem}[section]
\newtheorem{problem}{Problem}
\newtheorem{proposition}{Proposition}[section]
\newtheorem{lemma}{Lemma}[section]
\newtheorem{corollary}[theorem]{Corollary}
\newtheorem{example}{Example}[section]
\newtheorem{definition}[problem]{Definition}
\newcommand{\BEQA}{\begin{eqnarray}}
\newcommand{\EEQA}{\end{eqnarray}}
\newcommand{\define}{\stackrel{\triangle}{=}}
\theoremstyle{remark}
\newtheorem{rem}{Remark}
\begin{document}

\bibliographystyle{IEEEtran}
\vspace{3cm}

\title{Assignment-9}
\author{EE224BTECH11044 - Muthyala koushik
}
\maketitle
\bigskip

\renewcommand{\thefigure}{\theenumi}
\renewcommand{\thetable}{\theenumi}

\section{2018-EE 40-52}
\begin{enumerate}[start=40]
	\item The Fourier transform of a continuous-time signal $x\brak{t}$ is given by $X\brak{\omega}=\frac{1}{\brak{10+j\omega}^2}$, $-\infty<\omega<\infty$, where $j=\sqrt{-1}$ and $\omega$ denotes frequency. Then the value of $\abs{\ln{x\brak{t}}}$ at $t=1$ is $\rule{2cm}{0.4pt}$ \brak{\text{up to 1 decimal place}}. \brak{\text{$\ln$ denotes the logarithm to base $e$}}\\

	\item In the circuit shown in the figure, the bipolar junction transistor $\brak{\text{BJT}}$ has a current gain $\beta = 100$. The base-emitter voltage drop is a constant, $V_{BE} = 0.7 V$. The value of the Thevenin equivalent resistance $R_{Th}$ $\brak{\text{in \ohm}}$ as shown in the figure is $\rule{2cm}{0.4pt}$ $\brak{\text{up to 2 decimal places}}$.\\
\begin{figure}[H]
    \centering
    \begin{circuitikz}
\tikzstyle{every node}=[font=\large]

\draw (8.5,14) to[Tnpn, transistors/scale=1.19] (6.5,14);
\draw (4.5,14) to[american voltage source] (4.5,9.75);
\draw (7.5,11.5) to[american voltage source] (7.5,9.75);
\draw (4.5,14) to[R] (6.5,14);
\draw (7.5,13.25) to[R] (7.5,11.5);
\draw (4.5,9.75) to[short] (6.5,9.75);
\draw (4.5,9.75) to[short] (11.75,9.75);
\draw (10,14) to[R] (10,9.75);
\draw (8.25,14) to[short] (11.75,14);
\node [font=\large] at (12,14) {a};
\node [font=\large] at (12,9.75) {b};
\node [font=\large] at (5.5,14.5) {10 $\Omega$};
\node [font=\large] at (6.5,12.5) {10 k$\Omega$};
\node [font=\large] at (3.5,11.75) {15 V};
\node [font=\large] at (6.25,10.5) {10.7 V};
\node [font=\large] at (9.25,12) {1 k$\Omega$};
\node at (7.5,9.75) [circ] {};
\node at (10,14) [circ] {};
\node at (10,9.75) [circ] {};
\node at (11.75,9.75) [circ] {};
\node at (11.75,14) [circ] {};
\draw [->, >=Stealth] (12,12) -- (11,12);
\node [font=\large] at (12.5,12) {$R_{TH}$};
\end{circuitikz}
 % Includes the circuit diagram
\end{figure}
		

	\item As shown in the figure, $C$ is the arc from the point $\brak{3,0}$ to the point $\brak{0,3}$ on the circle $x^2 + y^2 = 9$. The value of the integral $\int_{C} \brak{y^2+2yx}dx + \brak{2xy+x^2}dy$ is $\rule{2cm}{0.4pt}$ $\brak{\text{up to 2 decimal places}}$.\\
\begin{figure}[H]
    \centering
    \begin{circuitikz}
\tikzstyle{every node}=[font=\large]
\draw  (6.75,15.25) circle (0.5cm);
\draw [->, >=Stealth] (4.75,15.25) -- (6.25,15.25);
\draw [->, >=Stealth] (7.25,15.25) -- (9,15.25);
\draw [->, >=Stealth] (6.75,13.25) -- (6.75,14.75);
\draw [short] (6.75,13.25) -- (12.25,13.25);
\draw  (9,16) rectangle (11,14.5);
\draw [->, >=Stealth] (11,15.25) -- (14,15.25);
\draw [short] (12.25,15.25) -- (12.25,13.25);
\draw [short] (4.75,16) -- (5.25,15.5)node[pos=0.5, fill=white]{R(s)};
\node [font=\large] at (6.25,15.75) {+};
\node [font=\large] at (7.25,14.75) {-};
\node [font=\large] at (10,15.25) {G(s)};
\node [font=\large] at (13.5,15.75) {C(s)};
\end{circuitikz}
 % Includes the circuit diagram
\end{figure}
		

	\item Let $f\brak{x}=3x^3-7x^2+5x+6$. The maximum value of $f\brak{x}$ over the interval $\sbrak{0, 2}$ is $\rule{2cm}{0.4pt}$ $\brak{\text{up to 1 decimal place}}$.\\

	\item Let $A = \begin{bmatrix} 
	1 & 0 & -1 \\
	-1 & 2 & 0 \\
	0 & 0 & -2
	\end{bmatrix}$ and $B = A^3 - A^2 - 4A + 5I$, where $I$ is the $3\times3$ identity matrix. The determinant of B is $\rule{2cm}{0.4pt}$ (up to 1 decimal place).\\

        \item The capacitance of an air-filled parallel-plate capacitor is $60$ pF. When a dielectric slab whose thickness is half the distance between the plates, is placed on one of the plates covering it entirely, the capacitance becomes $86$ pF. Neglecting the fringing effects, the relative permittivity of the dielectric is $\rule{2cm}{0.4pt}$ (up to 2 decimal places).\\

	\item he unit step response $y\brak{t}$ of a unity feedback system with open loop transfer function $G\brak{s}H\brak{s}=\frac{K}{\brak{s+1}^2\brak{s+2}}$ is shown in the figure. The value of K is $\rule{2cm}{0.4pt}$ $\brak{\text{up to 2 decimal places}}$.\\
\begin{figure}[H]
    \centering
    \begin{circuitikz}
\tikzstyle{every node}=[font=\normalsize]
\draw (6.5,18.5) to[short] (6.5,10.25);
\draw (5.5,11) to[short] (19.25,11);
\node [font=\LARGE] at (6.5,17.75) {-};
\node [font=\LARGE] at (6.5,16.75) {-};
\node [font=\LARGE] at (6.5,15.75) {-};
\node [font=\LARGE] at (6.5,14.75) {-};
\node [font=\LARGE] at (6.5,13.75) {-};
\node [font=\LARGE] at (6.5,12.75) {-};
\node [font=\LARGE] at (6.5,11.75) {-};
\node [font=\LARGE, rotate around={-90:(0,0)}] at (7.75,11) {-};
\node [font=\LARGE, rotate around={-90:(0,0)}] at (10.25,11) {-};
\node [font=\LARGE, rotate around={-90:(0,0)}] at (9,11) {-};
\node [font=\LARGE, rotate around={-90:(0,0)}] at (15.25,11) {-};
\node [font=\LARGE, rotate around={-90:(0,0)}] at (14,11) {-};
\node [font=\LARGE, rotate around={-90:(0,0)}] at (12.75,11) {-};
\node [font=\LARGE, rotate around={-90:(0,0)}] at (11.5,11) {-};
\node [font=\LARGE, rotate around={-90:(0,0)}] at (16.5,11) {-};
\node [font=\LARGE, rotate around={-90:(0,0)}] at (17.75,11) {-};
\node [font=\LARGE, rotate around={-90:(0,0)}] at (18.75,11) {-};
\draw [dashed] (6.5,14.75) -- (19.25,14.75);
\draw [short] (6.5,11) .. controls (7,10.5) and (7.25,21) .. (8.5,14.75);
\draw [short] (8.5,14.75) .. controls (9.25,11.75) and (9.5,16.75) .. (11,14.75);
\draw [short] (11,14.75) .. controls (12,13.5) and (12.25,15.5) .. (13.25,14.75);
\draw [short] (13.25,14.75) .. controls (14.75,14.5) and (14.75,15) .. (15.75,14.75);
\draw [short] (15.75,14.75) .. controls (17.25,14.5) and (17.5,15) .. (18.75,14.75);
\node [font=\LARGE] at (4.25,16.75) {y(t)};
\node [font=\LARGE] at (18,9.75) {t (sec)};
\node [font=\normalsize] at (7.75,10.5) {2};
\node [font=\normalsize] at (9,10.5) {4};
\node [font=\normalsize] at (10.25,10.5) {6};
\node [font=\normalsize] at (11.5,10.5) {8};
\node [font=\normalsize] at (12.75,10.5) {10};
\node [font=\normalsize] at (14,10.5) {12};
\node [font=\normalsize] at (15.25,10.5) {14};
\node [font=\normalsize] at (16.5,10.5) {16};
\node [font=\normalsize] at (17.75,10.5) {18};
\node [font=\normalsize] at (19,10.5) {20};
\node [font=\normalsize] at (6,11.75) {0.2};
\node [font=\normalsize] at (6,12.75) {0.4};
\node [font=\normalsize] at (6,13.75) {0.6};
\node [font=\normalsize] at (6,14.75) {0.8};
\node [font=\normalsize] at (6.25,15.75) {1};
\node [font=\normalsize] at (6,16.75) {1.2};
\node [font=\normalsize] at (6,17.75) {1.4};
\draw [short] (18.75,14.75) -- (19,14.75);
\end{circuitikz}
 % Includes the circuit diagram
\end{figure}
		


	\item A three-phase load is connected to a three-phase balanced supply as shown in the figure. If $V_{an}=100\angle0^\circ,V_{bn}=100\angle-120^\circ$ and $V_{cn}=100\angle-240^\circ$ $\brak{\text{angles are considered positive in the anti-clockwise direction}}$, the value of R for zero current in the neutral wire is $\rule{2cm}{0.4pt}$ $\brak{\text{up to 2 decimal places}}$.\\
\begin{figure}[H]
    \centering
    \begin{circuitikz}
\tikzstyle{every node}=[font=\large]
\draw  (6.75,15.25) circle (0.5cm);
\draw [->, >=Stealth] (4.75,15.25) -- (6.25,15.25);
\draw [->, >=Stealth] (7.25,15.25) -- (9,15.25);
\draw [->, >=Stealth] (6.75,13.25) -- (6.75,14.75);
\draw [short] (6.75,13.25) -- (12.25,13.25);
\draw  (9,16) rectangle (11,14.5);
\draw [->, >=Stealth] (11,15.25) -- (14,15.25);
\draw [short] (12.25,15.25) -- (12.25,13.25);
\draw [short] (4.75,16) -- (5.25,15.5)node[pos=0.5, fill=white]{R(s)};
\node [font=\large] at (6.25,15.75) {+};
\node [font=\large] at (7.25,14.75) {-};
\node [font=\large] at (10,15.25) {G(s)};
\node [font=\large] at (13.5,15.75) {C(s)};
\end{circuitikz}
 % Includes the circuit diagram
\end{figure}
		

	\item The voltage across the circuit in the figure, and the current through it, are given by the following expressions:\\$v\brak{t}=5-10\cos{\brak{\omega t+60^\circ}}V$\\$i\brak{t}=5+X\cos{\brak{\omega t}}A$\\
	where $\omega =100\pi$ radian/s. If the average power delivered to the circuit is zero, then the value of $X\brak{\text{in Ampere}}$ is $\rule{2cm}{0.4pt}$ $\brak{\text{up to 2 decimal places}}$.\\
\begin{figure}[H]
    \centering
    \begin{circuitikz}
\tikzstyle{every node}=[font=\large]
\draw  (6.75,15.25) circle (0.5cm);
\draw [->, >=Stealth] (4.75,15.25) -- (6.25,15.25);
\draw [->, >=Stealth] (7.25,15.25) -- (8.75,15.25);
\draw [->, >=Stealth] (6.75,13.25) -- (6.75,14.75);
\draw [short] (6.75,13.25) -- (15.25,13.25);
\draw  (12,16) rectangle (14,14.5);
\draw [->, >=Stealth] (14,15.25) -- (17,15.25);
\draw [short] (15.25,15.25) -- (15.25,13.25);
\draw [short] (4.75,16) -- (5.25,15.5)node[pos=0.5, fill=white]{R(s)};
\node [font=\large] at (6.25,15.75) {+};
\node [font=\large] at (7.25,14.75) {-};
\node [font=\large] at (13,15.25) {G(s)};
\node [font=\large] at (16.25,15.75) {C(s)};
\draw [->, >=Stealth] (10.25,15.25) -- (12,15.25);
\draw  (8.75,16.25) rectangle  node {\large $\frac{K_I}{s}$} (10.25,14.25);
\end{circuitikz}
 % Includes the circuit diagram
\end{figure}
		

        \item A phase controlled single phase rectifier, supplied by an AC source, feeds power to an R-L-E load as shown in the figure. The rectifier output voltage has an average value given by $V_0=\frac{v_m}{2\pi}\brak{3+\cos{\alpha}}$, where $V_m=80\pi$ volts and $\alpha$ is the firing angle. If the power delivered to the lossless battery is $1600$ W, $\alpha$ in degree is $\rule{2cm}{0.4pt}$ (up to $2$ decimal places).\\
\begin{figure}[H]
    \centering
    \begin{circuitikz}
\tikzstyle{every node}=[font=\Large]
\draw (8,15.25) to[sinusoidal voltage source, sources/symbol/rotate=auto] (8,10.75);
\draw (8,15.25) to[short] (9.25,15.25);
\draw (8,10.75) to[short] (9.25,10.75);
\draw  (9.25,16) rectangle (11,10);
\draw (11,15.25) to[short] (11.5,15.25);
\draw (11,10.75) to[short] (11.5,10.75);
\draw (13,10.75) to[short, -o] (11.5,10.75) ;
\draw (13,15.25) to[short, -o] (11.5,15.25) ;
\draw (13,15.25) to[R] (13,13.5);
\draw (13,13.5) to[L ] (13,12);
\draw [->, >=Stealth] (12,11.5) -- (12,14.5);
\draw (10.25,12.25) to[D] (10.25,13.75);
\draw (10.25,13.25) to[short, -o] (10.75,13.75) ;
\draw [short] (12.75,11.25) -- (13.25,11.25);
\draw [short] (12.5,11.5) -- (13.5,11.5);
\draw [short] (13,12.25) -- (13,11.5);
\draw [short] (13,11.25) -- (13,10.75);
\node [font=\large] at (13.5,11.75) {+};
\node [font=\large] at (13.5,11) {-};
\node [font=\normalsize] at (14.25,11.75) {80 V};
\node [font=\normalsize] at (14.25,11.25) {Battery};
\node [font=\normalsize] at (14,12.75) {10 mH};
\node [font=\normalsize] at (13.5,14.5) {2 $\Omega$};
\node [font=\normalsize] at (11.75,15.5) {+};
\node [font=\Large] at (11.75,10.5) {-};
\node [font=\normalsize] at (11.5,13) {$V_0$};
	\node [font=\Large] at (6,13) {$V_m \sin{\brak{\omega t}}$};
\end{circuitikz}
 % Includes the circuit diagram
\end{figure}
		

	\item The figure shows two buck converters connected in parallel. The common input dc voltage for the converters has a value of $100$ V. The converters have inductors of identical value. The load resistance is $1\ohm$ The capacitor voltage has negligible ripple. Both converters operate in the continuous conduction mode. The switching frequency is $1$ kHz, and the switch control signals are as shown. The circuit operates in the steady state. Assuming that the converters share the load equally, the average value of $i_{S1}$, the current of switch $S1$ (in Ampere), is $\rule{2cm}{0.4pt}$ (up to $2$ decimal places).\\
\begin{figure}[H]
    \centering
    \begin{circuitikz}
\tikzstyle{every node}=[font=\large]
\draw (7,11.25) to (7,10.75) node[ground]{};
\draw (5.5,13.5) to (5.5,13) node[ground]{};
\draw (10.5,11.25) to (10.5,10.75) node[ground]{};
\draw (15.25,12.25) to (15.25,11.75) node[ground]{};
\draw (5.5,13.5) to[sinusoidal voltage source, sources/symbol/rotate=auto] (5.5,15.25);
\draw (5.5,15.25) to[R] (9.25,15.25);
\draw (9.25,15.25) to[R] (14,15.25);
\draw (9.25,15.25) to[short] (9.25,13.75);
\draw (12.5,13.25) node[op amp,scale=1] (opamp2) {};
\draw (opamp2.+) to[short] (11,12.75);
\draw  (opamp2.-) to[short] (11,13.75);
\draw (13.7,13.25) to[short](14,13.25);
\draw (9.25,13.75) to[short] (11.75,13.75);
\draw (7,11.25) to[sinusoidal voltage source, sources/symbol/rotate=auto] (7,12.75);
\draw (7,12.75) to[R] (11,12.75);
\draw (10.5,12.75) to[R] (10.5,10.75);
\draw (14,13.25) to[short] (15.25,13.25);
\draw (14,15.25) to[short] (14,13.25);
\node [font=\large] at (15.5,13.25) {+};
\node [font=\large] at (15.5,12.25) {-};
\node [font=\large] at (6.5,12.5) {+};
\node [font=\large] at (6.5,11.5) {-};
\node [font=\large] at (5,14.75) {+};
\node [font=\large] at (5,14) {-};
\node [font=\large] at (4.25,14.5) {$v_{in1}$};
\node [font=\large] at (6,12) {$v_{in2}$};
\node [font=\large] at (7.5,15.75) {10.5 k$\Omega$};
\node [font=\large] at (11.75,15.75) {101 k$\Omega$};
\node [font=\large] at (8.75,13.5) {9.5 k$\Omega$};
\node [font=\large] at (11.75,11.75) {101.5 k$\Omega$};
\node [font=\large] at (16,13) {$v_0$};
\node [font=\large] at (2.75,6.5) {};
\end{circuitikz}
 % Includes the circuit diagram
\end{figure}
		

	\item A $3$-phase $900$ kVA, $3$ kV /$\sqrt{3}$ kV ($\Delta$/Y), $50$ Hz transformer has primary (high voltage side) resistance per phase of $0.3 \ohm$ and secondary (low voltage side) resistance per phase of $0.02 \ohm$. Iron loss of the transformer is $10$ kW. The full load $\%$ efficiency of the transformer operated at unity power factor is $\rule{2cm}{0.4pt}$ (up to $2$ decimal places).\\

	\item A $200$ V DC series motor, when operating from rated voltage while driving a certain load, draws $10$ A current and runs at $1000$ r.p.m. The total series resistance is $1\ohm$. The magnetic circuit is assumed to be linear. At the same supply voltage, the load torque is increased by $44\%$. The speed of the motor in r.p.m. (rounded to the nearest integer) is $\rule{2cm}{0.4pt}$ .



\end{enumerate}

\end{document}
