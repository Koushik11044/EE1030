%iffalse
\let\negmedspace\undefined
\let\negthickspace\undefined
\documentclass[journal,12pt,twocolumn]{IEEEtran}
\usepackage{cite}
\usepackage{amsmath,amssymb,amsfonts,amsthm}
\usepackage{algorithmic}
\usepackage{graphicx}
\usepackage{textcomp}
\usepackage{xcolor}
\usepackage{txfonts}
\usepackage{listings}
\usepackage{enumitem}
\usepackage{mathtools}
\usepackage{gensymb}
\usepackage{comment}
\usepackage[breaklinks=true]{hyperref}
\usepackage{tkz-euclide} 
\usepackage{listings}
\usepackage{gvv}                                        
\def\inputGnumericTable{}                                 
\usepackage[latin1]{inputenc}                                
\usepackage{color}                                            
\usepackage{array}                                             
\usepackage{longtable}                                       
\usepackage{calc}                                             
\usepackage{multirow}                                         
\usepackage{hhline}                                           
\usepackage{ifthen}                                           
\usepackage{lscape}
\usepackage{multicol}

\newtheorem{theorem}{Theorem}[section]
\newtheorem{problem}{Problem}
\newtheorem{proposition}{Proposition}[section]
\newtheorem{lemma}{Lemma}[section]
\newtheorem{corollary}[theorem]{Corollary}
\newtheorem{example}{Example}[section]
\newtheorem{definition}[problem]{Definition}
\newcommand{\BEQA}{\begin{eqnarray}}
\newcommand{\EEQA}{\end{eqnarray}}
\newcommand{\define}{\stackrel{\triangle}{=}}
\theoremstyle{remark}
\newtheorem{rem}{Remark}
\begin{document}

\bibliographystyle{IEEEtran}
\vspace{3cm}

\title{Assignment-1}
\author{EE224BTECH11044 - Muthyala koushik$^{*}$% <-this % stops a space
}
\maketitle
\newpage
\bigskip

\renewcommand{\thefigure}{\theenumi}
\renewcommand{\thetable}{\theenumi}

\textbf{1.Section-B : JEE Main/AIEEE}
\begin{enumerate}[start=4]
	\item If $f: \mathbb{R} \to \mathbb{R}$ satisfies $f\brak{x+y} = f\brak{x} + f\brak{y}$ for all $x, y \in \mathbb{R}$ and $f(1) = 7$, then $\sum_{r=1}^n f(r)$ is \hfill(2003)
		
    \begin{enumerate}
        \item $\frac{7n(n+1)}{2}$\\
        \item $\frac{7n}{2}$\\
        \item $\frac{7(n+1)}{2}$\\
        \item $7n + (n+1)$
    \end{enumerate}

    \item The function $f$ from the set of natural numbers to integers is defined by \hfill(2003)
               $f(n) = \begin{cases} \frac{n-1}{2} & \text{when } n \text{ is odd}, \\-\frac{n}{2} & \text{when } n \text{ is even}.\end{cases}$ is

	\begin{enumerate}
		\item neither one-one nor onto
		\item one-one but not onto
		\item onto but not one-one
		\item one-one and onto both.
	\end{enumerate} 

\item The range of the function $f(x)=\nPr{7-x}{x-3}$ is \hfill(2004)
	\begin{enumerate}
		\item $\cbrak{1,2,3,4,5}$
		\item $\cbrak{1,2,3,4,5,6}$
		\item $\cbrak{1,2,3,4}$
		\item $\cbrak{1,2,3}$
	\end{enumerate}

\item If $f:R\to S$, defined by $f(x)=\sin{x}-\sqrt{3}\cos{x}+1$,is onto,then the interval of S is \hfill(2004)

	       \begin{enumerate}
		       \item $\sbrak{-1,3}$
		       \item $\sbrak{-1,1}$
		       \item $\sbrak{0,1}$
		       \item $\sbrak{0,3}$
	       \end{enumerate}

       \item The graph of the function $y=f(x)$ is symmertrical about the line $x=2$,then \hfill(2004)
	       \begin{enumerate}
		       \item $f(x)=-f(-x)$
		       \item $f(2+x)=f(2-x)$
		       \item $f(x)=f(-x)$
		       \item $f(x+2)=f(x-2)$
	       \end{enumerate}

       \item The domain of the function $f(x)=\frac{\sin^{-1}\brak{x-3}}{\sqrt{9-x^{2}}}$ is \hfill(2004)
              \begin{enumerate}
		      \item $[1,3]$
		      \item $[2,3)$
		      \item $[1,2]$
		      \item $[2,3]$
	      \end{enumerate}

      \item Let $f:(-1,1) \to B$,be a function defined by  $f(x)=\tan^{-1}\frac{2x}{1-x^{2}}$,then $f$ is both one-one and onto when $B$ is the interval \hfill(2005)
             \begin{enumerate}
		     \item $\brak{0,\frac{\pi}{2}}$ \\
		     \item $\lsbrak{0},\rbrak{\frac{\pi}{2}}$ \\
		     \item $\sbrak{-\frac{\pi}{2},\frac{\pi}{2}}$ \\
		     \item $\brak{-\frac{\pi}{2},\frac{\pi}{2}}$
	     \end{enumerate}
     \item A function is macthed below against an interval where it is supposed to increasing.Which of the following pairs is incorrectly matcher? \hfill(2005)
	     \begin{multicols}{2}
		     \textbf{Interval}
                     \begin{enumerate}
                        \item $(-\infty,\infty)$
                        \item $[2,\infty)$
			\item $\lbrak{-\infty},\rsbrak{\frac{1}{3}}$
                        \item $(-\infty,-4)$
                     \end{enumerate}

                  \columnbreak

                    \textbf{Function}\\
                      $x^{3} - 3x^{2} + 3x + 3$\\
                      $2x^{3} - 3x^{2} + 3x + 3$\\
                      $3x^{2} - 2x + 1$\\
                      $x^{3} + 6x^{2} + 6$\\
	     \end{multicols}

     \item A real valued function $f(x)$ satisfies the functional equation $$f(x-y)=f(x)f(y)-f(a-x)f(a+y)$$\\
	     where a given constant and $f(0)=1$,$f(2a-x)$ is equal to \hfill(2005)
	     \begin{enumerate}
		     \item $-f(x)$
		     \item $f(x)$
		     \item $f(a)+f(a-x)$
		     \item $f(-x)$
	     \end{enumerate}

     \item The Largest interval lying in $\brak{-\frac{\pi}{2},\frac{\pi}{2}}$for which the function,
	     $f(x)=4^{-x^{2}}+\cos^{-1}\brak{\frac{x}{2}-1}+\log(\cos{x})$,is defined,is \hfill(2007)
	     \begin{enumerate}
		     \item $\lsbrak{-\frac{\pi}{4}},\rbrak{\frac{\pi}{2}}$ \\
		     \item $\lsbrak{0},\rbrak{\frac{\pi}{2}}$ \\
		     \item $[0,\pi]$ \\
		     \item $\brak{-\frac{\pi}{2},\frac{\pi}{2}}$
	     \end{enumerate}
     \item Let $f:N\to Y$ be a function defined as $f(x)=4x+3$ where Y=$\cbrak{y\in \mathbb{N}:y=4x+3 for some x\in \mathbb{N}}$.Show that $f$ is invertible and its inverse is \hfill(2008)
	     \begin{enumerate}
		     \item $g(y)=\frac{3y+4}{3}$ \\
		     \item $g(y)=4+\frac{y+3}{4}$ \\
		     \item $g(y)=\frac{y+3}{4}$ \\
		     \item $g(y)=\frac{y-3}{4}$
	     \end{enumerate}
     \item Let $f(x)=(x+1)^{2}-1,x\leq-1$\\
	     \textbf{Statement-1:}The set $\cbrak{x:f(x)=f^{-1}(x)=\cbrak{0,-1}}$\\
	     \textbf{Statement-2:}$f$ is a bijection.\hfill(2009)
	     \begin{enumerate}
		     \item Statement-1 is true,Statement-2 is true.Statement-2 is  not a correct explanation for Statement-1.
		     \item Statement-1 is true,Statement-2 is flase.
		     \item Statement-1 is false,Statement-2 is true.
		     \item Statement-1 is true,Statement-2 is true.Statement-2 is not a correct explanation for Statement-1.
	     \end{enumerate}
     \item For real $x$,let $f(x)=x^{3}+5x+1$,then \hfill(2009)
	     \begin{enumerate}
		     \item $f$ is onto $\mathbb{R}$ but not one-one
		     \item $f$ is one-one and onto $\mathbb{R}$
		     \item $f$ is neither one-one nor onto $\mathbb{R}$
		     \item $f$ is one-one but not onto $\mathbb{R}$
	     \end{enumerate}
     \item The domain of the function $f(x)=\frac{1}{\sqrt{\abs{x}-x}}$ is \hfill(2011)\\
	     \begin{enumerate}
		     \item $(0,\infty)$
		     \item $(-\infty,0)$
		     \item $(-\infty,\infty)-\cbrak{0} $
		     \item $(-\infty,\infty)$
	     \end{enumerate}

     \item For $x\in \mathbb{R}-\{0,1\}$,let $f_1(x)=\frac{1}{x}$,$f_2(x)=1-x$ and $f_3(x)=\frac{1}{1-x}$ be the three given functions.If a function,$J(X)$ satisfies ($f_2oJof_1$)$(x)=f_3(x)$ then J(x) is equal to: \hfill(JEE M 2019-9 Jan(M))
	     \begin{enumerate}
		     \item $f_3(x)$
		     \item $f_3(x)$
		     \item $f_2(x)$
		     \item $f_1(x)$
	     \end{enumerate}
	
\end{enumerate}


\end{document}
